\documentclass[12pt]{article}
%\usepackage[a4paper]{geometry}
\usepackage[a4paper,left=1in,right=1in,top=1in,bottom=1in]{geometry}
\usepackage[italian,english]{babel}
\usepackage{amssymb}
\usepackage{amsmath}  %\documentstyle{numline} 
\usepackage{latexsym}
\usepackage{graphicx}
\usepackage{setspace}
\usepackage{rotating}
\usepackage{multirow}
\usepackage{verbatim}
\usepackage{subfigure}
\usepackage{authblk}
%\usepackage{fancyhdr}
\usepackage[usenames]{color}
\usepackage{soul}
\usepackage{setspace}
\usepackage{threeparttable}
\usepackage{caption}
\usepackage{dcolumn}
%\newcolumntype{.}{D{.}{.}{-1}}
\usepackage{placeins}
\usepackage{booktabs}
\usepackage{color}

\usepackage[colorlinks=true, linkcolor=true, urlcolor=blue, allcolors=blue]{hyperref}  % for buttons


%   \onehalfspacing
\sethlcolor{Orange}
%\usepackage[first,bottom]{draftcopy}
%\usepackage{draftwatermark}
%\SetWatermarkText{Draft Version}
%\SetWatermarkScale{3}
\usepackage{eurosym}
%%%%  incompatible with biblatex  \usepackage[authoryear]{natbib}
\usepackage[authoryear]{natbib}
\bibliographystyle{myplainnat}
\usepackage{url}
%\usepackage{doi}
% \bibliographystyle{dcu}%{agsm}  %
\graphicspath{{./figures/}}


% \usepackage[style=authoryear-comp,backend=bibtex,uniquename=init,giveninits,maxcitenames=2,natbib=true,url=true,doi=false,eprint=false]{biblatex}
% \bibliography{bib_file_esempio.bib}
% % \bibliography{/home/marco/work/bibliog/bib_file_cvs_synced}
% %\setcolor*{bibliography entry title}{fg=blue}
% %\setcolor*{bibliography entry author}{fg=black}
% %\setcolor*{bibliography entry location}{fg=black}
% %\setcolor*{bibliography entry note}{fg=black}
% %\setbeamertemplate{bibliography item}{}
% \renewcommand*{\bibfont}{\small}
% \setlength\bibitemsep{1.5\itemsep}

% \newcommand\mkbibcolor[2]{\textcolor{#1}{\hypersetup{citecolor=#1}#2}}
% \DeclareCiteCommand{\cite}[\mkbibcolor{blue}]
%   {\usebibmacro{prenote}}%
%   {\usebibmacro{citeindex}%
%    \usebibmacro{cite}}
%   {\multicitedelim}
%   {\usebibmacro{postnote}}


\renewcommand{\refname}{Bibliografia}

\hyphenation{in-tro-du-zio-ne     in-te-res-sa-te   in-te-res-se   rap-pre-sen-ta-to   mo-del-li   mo-del-lo   si-ste-ma-ti-co   sem-pli-ci-sti-co   o-pe-ra-to-ri   me-de-si-me   i-ta-lia-no  ve-ri-fi-ca   sta-ti-sti-che   sta-ti-sti-ca-men-te   nor-ma-liz-za-re   con-si-de-ra-re   in-clu-sio-ne   e-sclu-sio-ne   e-sclu-de-re  o-pe-ra-zio-ni   o-pe-ra-zio-ne   sta-ti-sti-ca   ri-fe-ri-men-to   se-con-do  Se-con-do  de-via-zio-ne   va-rian-za  ela-sti-ca   con-si-de-ra-zio-ni mo-di-fi-che ma-nu-factu-ring Ma-nu-factu-rin Pro-dut-ti-vi-ta pro-dut-ti-vi-ta in-du-stria cam-bia-men-to am-mi-ni-stra-zio-ne}




\newcommand{\EP}{\mathrm{ \scriptscriptstyle EP}}
\newcommand{\SEP}{\mathrm{ \scriptscriptstyle SEP}}
\newcommand{\AEP}{\mathrm{ \scriptscriptstyle AEP}}
\newcommand{\GHYP}{\mathrm{ \scriptscriptstyle GHYP}}
\newcommand{\ASY}{\mathrm{ \scriptscriptstyle ASY}}
\newcommand{\RMSE}{\mathrm{ \scriptscriptstyle RMSE}}

\bibpunct{(}{)}{;}{a}{,}{,}


\title{ \textsc{Note per il lavoro di tesi}\thanks{Versione v1.1,
    Ottobre 2023. Ringrazio gli studenti che negli anni e nelle varie
    sedi con le loro domande hanno contribuito all stesura di queste
    pagine. Continuate ad inviare eventuali suggerimenti e contributi
    al mio indirizzo di lavoro. Anche segnalazioni su link non pi\`u
    validi o simili sono utili.}}


\author{Marco Grazzi\thanks{Universit\`a Cattolica del Sacro Cuore, Dipartimento di Politica Economica.}}

 % \small{\date{\today}}
\date{\empty}


\begin{document}
\maketitle





\section{Introduzione}\label{sec:intro}

Queste note sono rivolte agli studenti che stanno valutando di
scrivere la tesi sotto la mia supervisione. Leggetele con attenzione
prima di venire a ricevimento o contattarmi per mail. Queste
indicazioni si aggiungono (e non sostituiscono!) alle istruzioni
fornite dalle segreterie studenti.

Il presente documento si rivolge sia agli studenti di laurea triennale
(LT) che di laurea magistrale (LM). A questi ultimi (LM) anticipo che,
in virt\`u del lavoro svolto in classe durante il corso, sar\`a
richiesta una parte di analisi empirica, preferibilmente di tipo
microeconomico. A tal fine gli studenti possono utilizzare le fonti
disponibili nella biblioteca tra le risorse elettroniche, che trovate
al link:
\href{https://biblioteche.unicatt.it/brescia-servizi-per-utenti-interni-ricerche-bibliografiche-e-banche-dati}{Risorse
  Elettroniche Unicatt}. Cliccate poi su Banche dati. Verificate se
sia necessario autenticarsi con le vostre credenziali.

%\subsubsection*{Scelta della materia e dell'argomento di tesi}

La tesi dovr\`a riguardare argomenti economici, in particolare quelli
affrontati nel corso. Questo vuol dire, solo per fare un esempio, che
se volete studiare le strategie di un'impresa o affrontare un caso
aziendale, questa non \`e la materia adeguata.

Se invece siete interessati ad una materia economica, ed in
particolare ad uno dei temi affrontati nel mio corso, allora il mio
(caloroso) suggerimento \`e di proporre voi uno o pi\`u argomenti di
tesi; avrete cos\`i occasione di scegliere e sviluppare un argomento di
vostro interesse. Al primo ricevimento discuteremo le vostre proposte.



\section{I tempi}
Siete ovviamenti voi stessi i responsabili del rispetto dei tempi
richiesti dall'amministra-zione.\footnote{Qui trovate il link con le
  scadenze:
  \href{https://milano.unicatt.it/facolta/economia-informazioni-per-gli-studenti-esami-di-laurea-294}{link
    scadenze}. } Considerate che in certi periodi mi pu\`o capitare di
ricever molte richieste di tesi e non sempre riesco a dare seguito a
tutte. Mi scuso in anticipo per questo.



\section{Lavoro preliminare: le fonti}\label{sec:fonti}
Una tesi, sia LT che LM, prevede un lavoro preliminare sulle fonti
bibliografiche. Cosa \`e gi\`a stato detto su un certo argomento?
Suggerisco di partire dai materiali del corso (libro, appunti,
articoli). Ci sono poi innumerevoli strumenti informatici che
permettono di espandere il raggio d'azione e risalire a fonti
precedenti o, al contrario, di spostarsi a lavori successivi che
citano quello di partenza; tra i molti richiamo
\textit{google.scholar}. Nel campo di ricerca inserite il titolo del
lavoro che cercate o parole chiavi del vostro argomento di
ricerca. Tra le fonti individuate date preferenza a quelle pi\`u
recenti e che sono gi\`a pubblicate su rivista in quanto hanno gi\`a
superato un vaglio editoriale. Segnalo il portale
\href{https://voxeu.org/}{voxeu.org} che pubblica articoli scritti da
economisti per divulgare i risultati di lavori scientifici. Segnalo
inoltre inoltre le seguenti riviste scientifiche che hanno un
approccio rigoroso, ma hanno carattere pi\`u divulgativo e quindi
pi\`u facilmente accessibili:
\href{https://www.aeaweb.org/journals/jep}{Journal of Economic
  Perspective} e \href{https://www.aeaweb.org/journals/jel}{Journal of
  Economic Literature}. Per accedere agli articoli scientifici non
disponibili liberamente quando non siete fisicamente in Universit\`a,
utilizzate le risorse elettroniche di Ateneo (facendo login)
\href{https://www.unicatt.it/off-campus}{off-campus}.

Pu\`o essere utile perdersi per qualche giorno in questo ``mare'' di
fonti; \`e poi necessario concludere questa fase di esporazione e
delimitare il campo di analisi. Al termine di questo lavoro di
raccolta informazioni, selezione e lettura suggerisco di abbozzare la
struttura dell'elaborato.

Considerate infine che gran parte della bibliografia economica degli
anni pi\`u recenti \`e in lingua inglese.



\section{La struttura dell'elaborato}
In sintesi e semplificando, un elaborato deve contenere innanzitutto
un'introduzione in cui si illustra la questione (la tesi, la ``domanda
di ricerca'') che sar\`a affrontata. La stessa introduzione (o il
capitolo successivo) deve fornire i riferimenti alla letteratura
scientifica di riferimento (andate alle fonti, limitate al massimo o
escludete proprio i riferimenti alla stampa generalista). Per quanto
recente pu\`o essere l'argomento che avete scelto, qualcuno vi avr\`a
gi\`a anticipato: prima di tentare di aggiungere il vostro contributo
fornite quindi una vostra interpretazione dello \textit{stato
  dell'arte}.  I capitoli centrali sviluppano il tema. Le conclusioni
vi permettono di aggiungere il vostro personale contributo
interpretativo sull'argomento che avete scelto.


\subsection*{L'analisi empirica (LM)}
Se la vostra \`e una tesi di LM uno dei capitoli centrali sar\`a
dedicato all'analisi empirica di uno o pi\`u settori economici con
dati micro, ovvero a livello di impresa. I dati li potete reperire
tramite le risorse online Unicatt (vedere a
\href{https://biblioteche.unicatt.it/brescia-servizi-per-utenti-interni-ricerche-bibliografiche-e-banche-dati}{questo}
link). Se tramite canali personali o lavorativi avete accesso a fonti
che ritenete pi\`u interessanti, valuteremo questa possibilit\`a ad un
ricevimento. Deve per\`o esser chiaro che il lavoro di tesi nella mia
materia deve riguardare l'analisi di un settore industriale (o
comunque uno dei temi richiamati nella
Sezione~\ref{sec:idee-tesi}). Nella mia materia non sono quindi
adeguati dei \textit{case studies} focalizzati su un'impresa o su un
numero di imprese talmente ridotto da non risultare rappresentativo
dell'intero settore. A tal proposito anticipo che per affrontare
adeguatamente il lavoro empirico \`e \textit{necessario} ricorrere a
software che permettano la gestione di database di grandi dimensioni,
in quanto il numero di imprese (da qualche centinaio a qualche
migliaio), le variabili per le analisi, gli anni di osservazione
generano una mole di dati che non pu\`o essere gestita ed analizzati
con fogli di calcolo come Excel. Ciascuno studente pu\`o utilizzare il
software che preferisce. Tra i molti esistenti segnalo, senza pretesa
di essere esaustivo, \href{https://www.stata.com/}{Stata},
\href{https://www.r-project.org/}{R} \textit{a free software}, o la
possibilit\`a di gestire dati offerta da
\href{https://www.python.org/}{Python},
\href{http://www.data-analysis-in-python.org/}{data analysis in
  Python}.

Nel merito, il tipo di analisi empiriche pu\`o riguardare anche (in modo non
esaustivo) i temi gi\`a affrontati in classe. Qui potete trovare il
riferimento a questo
visto durante il corso:\\
\href{http://mgrazzi.github.io/istruz-lavoro-gruppo.pdf}{mgrazzi.github.io/istruz-lavoro-gruppo.pdf}.




\section{Sulle spalle dei giganti: citare le fonti}%Fonti bibliografiche}

Allo stesso modo con cui \`e importante fornire informazioni sul del
materiale utilizzato nell'elaborato (ad esempio, dati, tabelle,
figure), \`e necessario citare le fonti bibliografiche a cui si fa
riferimento. Tutte le fonti (articoli, libri, capitoli in libro,
working papers, etc) citate devono poi essere riportate nella
bibliografia in fondo alla tesi. Riporto di seguito un testo (in
inglese) che contiene citazioni a vari lavori, tutti poi riportati in
bibliografia. 

\vspace{0.3cm}

\small{An important stream of literature within industrial
  economics has for long been interested in assessing the contribution
  to employment creation stemming from the different firm-size
  classes. In this respect, at least since \cite{birch_1981}, small
  firms have been considered as a much relevant source of job
  creation. The increasing availability of firm level dataset has
  further contributed to foster research on the issue, starting from
  the seminal works of \cite{davis_haltiwanger_1992} and
  \cite{davis_etal_1996}. These studies represented a relevant
  advancement for the understanding of employment and industrial
  dynamics, in that they confirmed, by means of new methodological and
  empirical tools, that smaller firms are major players in terms of
  job churning, hence contributing both to employment creation and
  destruction \citep[among the others, also refer
  to][]{davis_haltiwanger_1995}.}


\section{Alcune idee per argomenti di tesi}\label{sec:idee-tesi}
Come gi\`a spiegato nella sezione~\ref{sec:intro} vi esorto a proporre
un argomento di vostro interesse, ovviamente all'interno del perimetro
della materia. Riporto di seguito alcune idee con i relativi spunti,
assolutamente parziali, bibliografici.\footnote{\`E certo pi\`u
  interessante seguire un'idea di tesi che parta dal vostro
  interesse. Tuttavia durante il periodo segnato da COVID-2019 (da
  Marzo 2020) ho aggiunto questa sezione per favorire il lavoro da
  remoto, in assenza di ricevimenti in presenza.} Partite da questi
spunti per cercare nuovo materiale secondo le indicazioni fornite
nella Sezione~\ref{sec:fonti}.

Per le LT suggerisco di considerare i punti da
\ref{sec:beyond-gdp}~a~\ref{sec:public-debt} (ma possono essere
valutati anche gli altri). Mentre per le LM esclusivamente i punti
successivi.

\subsection{Oltre il GDP (\textit{Beyond GDP})}\label{sec:beyond-gdp}

Un punto di partenza \`e il rapporto di Stiglitz, Sen e Fitoussi
\citep{stiglitz_etal_beyond-gdp}. Considerate inoltre il rapporto ISTAT \href{https://www.istat.it/it/archivio/rapporto+bes}{Benessere Equo e Sostenibile}; i rapporti OECD
\href{http://www.oecd.org/statistics/better-life-initiative.htm}{Better
  Life Initiative};
\href{http://www.oecd.org/social/for-good-measure-9789264307278-en.htm}{For
  Good Measure} e questi altri strumenti
\href{https://www.project-syndicate.org/commentary/new-metrics-of-wellbeing-not-just-gdp-by-joseph-e-stiglitz-2018-12}{new
  metrics of wellbeing}; \href{https://www8.gsb.columbia.edu/faculty/jstiglitz/sites/jstiglitz/files/The%20Measurement%20of%20Economic%20Performance%20and%20Social%20Progress.pdf}{Economic Performance and Social Progress}.

  
\subsection{Reddito di inclusione, cittadinanza, o \textit{Basic Income}}\label{sec:basic-income}

Tra gli altri, partite da \cite{toso_2016},
\cite{vanparijs_2017_basic-income}. Per una prospettiva storica fare
riferimento a \cite{orsi_2018}.


\subsection{Disuguaglianza}\label{sec:inequality}

Tra i molti altri altri, partite da: \cite{atkinson_2010_top,
  piketty_2014_capital, piketty_2015_inequality, atkinson_2015,
  milanovic_2011_worlds, milanovic_2016_global}. Molto utile la
possibile di visualizzare dati su \href{https://wid.world/}{World
  Inequality Database}. Sull'Italia in particolare, vedere, tra gli
altri \cite{pianta_franzini_2016, pianta_2012_nove}


\subsection{Relazione tra disuguaglianza ed efficienza (o crescita)}\label{sec:inequality-growth}

Come visto nel corso di Politica Economica, la tassazione, uno degli
strumenti di redistribuzione che favorisce la riduzione della
disuguaglianza di reddito, potrebbe distorcere il comportamento degli
individui (che ad esempio, a seguito di un aumento delle tasse,
decidono di offrire meno lavoro) e quindi compromettere altri
obiettivi, come efficienza e crescita. Dallo sviluppo di questo
ragionamento discende il supposto trade-off tra equit\`a ed efficienza
\cite{okun_1975}. Le analisi empiriche mostrano invece una storia
differente. Si vedano tra gli altri i lavori di Ostry e co-autori tra
cui \cite{ostry_etal_2014_redistribution, berg_etal_2018_JEG,
  ostry_etal_2019_confronting}. Si veda anche l'articolo sul blog
\href{https://voxeu.org/article/redistribution-inequality-and-sustainable-growth}{voxeu.org}.







\subsection{Debito pubblico e crescita, sostenibilit\`a del debito, moltiplicatori fiscali}\label{sec:public-debt}

Tra i molti altri, il libro della World Bank su
\href{https://www.worldbank.org/en/research/publication/waves-of-debt}{Global
  Waves of Debt}; la AEA presidential lecture di Blanchard su Debito
Pubblico e bassi tassi di interesse \citep{blanchard_2019} ed una
\href{https://voxeu.org/content/olivier-wonderland}{risposta}. Inoltre vedere
\href{https://voxeu.org/article/debt-and-financial-crises-will-history-repeat-itself}{Debt
  and financial crises};
\href{https://voxeu.org/article/stress-testing-eu-fiscal-framework}{Stress
  testing}; \href{
  https://www.imf.org/external/pubs/ft/sdn/2015/sdn1510.pdf}{When
  public debt should be reduced}

Sui moltiplicatori fiscali vedere \cite{spilimbergo_etal_2009_fiscal}   e sull'Italia il lavoro di De Nardis e Pappalardo per l'ufficio parlamentare di Bilancio \href{http://www.upbilancio.it/nota-di-lavoro-12018/}{(link al paper).}


\subsection{Start-up, Entrepreneurship and Industry Dynamics}\label{sec:start-up}

The following paragraph is an excerpt from
\cite{grazzi_moschella_2018}. In addition refer also to
\cite{santarelli_vivarelli_2007, vivarelli_2013_ICC}. \vspace{0.5cm}


In this respect, at least
since the work of \cite{birch_1981}, small firms have been considered,
especially by policy makers, as a relevant source of job creation. The
hypothesis according to which smaller firms grow more than bigger ones
would represent an apparent violation of the Gibrat's law \citep[see,
among the others,][]{sutton_1997, lotti_etal_2003, coad_2009}. The
increasing availability of firm level dataset has further fostered
research on the topic
\citep[see, in particular,][]{davis_haltiwanger_1992, davis_etal_1996}. These first
works, in addition to providing a methodological benchmark for future
studies, confirmed that smaller firms were major players in terms of
job churning, hence contributing both to employment creation and
destruction. More recent contributions, also
encompassing the role of firm age, challenge that
evidence.\footnote{\cite{evans_1987_a, evans_1987_b} provided an early
  attempt to look at the effect of age on the size-growth
  relationship. The author, employing firm-level data for
  U.S. manufacturing firms, reported an inverse relationship between
  firm growth and size (holding firm age constant) and between firm
  growth and age (holding firm size constant). However, as noted by
  Evans, the available data underrepresented smaller firms.} In
particular, \cite{haltiwanger_etal_2013}, using data from the Census
Bureau's Longitudinal Business Database (LBD), show that the negative
relationship between firm size and growth disappears and may even
reverse sign among the biggest firms.  This evidence casts doubt on
policy interventions aimed at small businesses but ignoring the role
of firm age.  Along similar lines, \cite{lawless_2014}, using survey
data on Irish firms, show that once accounting for firm age the
inverse relationship between growth and size declines very markedly
across different age categories.  The present paper contributes to
this literature by showing that the export status is a
crucial dimension to understand firm growth patterns together with
firm age and size. 
%Further, we also investigate how the age and the size of the
%firms jointly contribute to shape their growth.


\subsection{The financing of innovation}\label{sec:fin-innov}

Much related to the previous topic, the issue of financing innovative
activities has emerged as distinct field. The issue is even more
relevant for start-up, as they tend to be more financially constrained
than mature firms.

Here are some of the research questions that are typically addressed
in this field. What is the role of public funding in supporting long
term / high risk projects \citep[][]{breschi_etal_2021, santoleri_etal_2022}? Is there any
evidence of crowding-in thanks to publich funding, the so called
certification hypothesis, \citep[][]{lerner_2002}?

What is the specific evidence in our country with the so-called
Start-up act? \citep[][]{manaresi_etal_2020}.

What are the peculiar needs of the so-called deep-tech industry 
\citep[][]{nedayvoda_etal_2021financing}?

And the overall effects of the so-called killer acquisitions which
occur when ``incumbent firms [may] acquire innovative targets solely
to discontinue the target's innovation projects and preempt future
competition'' \citep[][]{cunningham_etal_2021}.



  
\subsection{Increasing mark-ups}\label{sec:increasing-markups}

Negli ultimi anni un'ulteriore preoccupazione ha animato il dibattito
relativo alle politiche industriali. Tale questione \`e relativa
all'apparente aumento nei mark-ups e nella concentrazione in diversi
settori, in particolare in quelli dove gli investimenti in ICT
(includendovi anche robotics, AI, e simili tecnologie) sono pi\`u
rilevanti. Un primo spunto di analisi, ulteriori riferimenti
bibliografici \`e disponibile in \cite{berry_etal_2019_JEP,
  basu_2019_JEP, syverson_2019_JEP}.




\subsection{Technology, Employment and Industry Dynamics}\label{sec:tech-employment}

The following three paragraphs are an excerpt from
\cite{domini_etal_2019}, to which I refer for a broader discussion. On
the same topic also refer to \cite{vivarelli_2007, domini_etal_2021,
  domini_etal_2022}. \vspace{.5cm}


Technology is presented in the policy debate either as a major threat
to employment -- reviving the concept of technological unemployment
--, or as the main driver of societal change.  Such mix of fear and
excitement can also be explained with the difficulty to catch-up with
a moving target: the speed of change is increasing, both across the
revolutions themselves (it took a century to go from the mechanical to
the electrical paradigms, and then another one to the ICT revolution,
but less than fifty years to reach the digital age) and within the
current one, as emerging technologies pop-up every day.

Assessing how such innovations affect employment has been at the
centre of economic debates, both in terms of the effects on the single
person, i.e. how the changing working conditions affect the life of
people, as well as on employment at a more aggregate level (Ricardo,
Marx and Keynes all have discussed technological unemployment, for a
recent review, see \citealp{piva_vivarelli_2017}). Yet the extent and
the manners through which Artificial Intelligence, robotisation, and
digital technologies more in general, are expected to impact on work
are much broader than in previous waves of innovations.  In the
contemporary economic scenario one can envisage at least two,
relatively new, challenges. On the one hand, the type of jobs affected
is much more diffused and difficult to identify. Previously, it was
mostly manual jobs that were at risk of being replaced by a
machine. Currently, all jobs that are rich in routine-intensive,
highly codified tasks are exposed to the risk of being replaced by a
machine \citep[see, for instance,][]{autor_etal_2003_QJE,
  goos_etal_2014, autor_2015_JEP}. Moreover, this process is largely
orthogonal to the traditional classification in blue versus white
collar jobs \citep[among the others, refer to][]{frey_osborne_2017,
  trajtenberg_2018, furman_seamans_2018}.\footnote{In this respect,
  the distinction between codified and tacit knowledge, and its
  implication, as put forth in a vivid way by \cite{polanyi_1967}, has
  been very relevant in shaping the debate around the so-called
  Skill-Biased Technical Change \citep[see among the many
  others][]{autor_etal_2003_QJE, autor_2015}.} On the other hand, also
the share of firms and sectors that are, at least potentially, exposed
is much broad. Previous changes in technological paradigms have led to
structural change and reallocation across sectors and firms, to the
benefit of those able to reach the highest levels of productivity
growth. The digital revolution, together with the globalisation of
exchanges, instead requires all firms to rethink their production
process so as to respond to higher levels of complexity and
adaptability \citep{caliendo2012}.

%general literature gap
To date, most of the evidence on the effects of innovation on
employment relies either upon indirect measure of occupations that can
be impacted upon by technological progress (see, for example, the
routine task intensity index approached used, among the others, by
\citealp{autor_etal_2013_AER} and \citealp{goos_etal_2014}) or on
measures of technological adoption related to the ICT services (as in
\citealp{harrigan_etal_2016_NBER}).  On the contrary, evidence on the
direct effect of the most recent wave of automation technologies is
more scant.  \cite{acemoglu_restrepo_2017} find a negative effect of
robots adoption on employment across commuting zones in US during the
period 1990-2007 whereas in \cite{graetz_michaels_2018} robots are not
found to decrease employment in a sample of countries and industries
during the same period.  Both papers use aggregate data. Evidence at
the firm-level, where decisions about technological adoption are made,
is even more scarce.  \cite{bessen_2019} study a sample of Dutch firms
over the period 2000-2016 to show that automation increases the
probability of workers separating from their employers.


\bigskip
\subsection*{Quale programma utilizzare: la mia scelta}
Questo documento \`e stato scritto con \LaTeX. Ai seguenti link
trovate il file \textit{sorgente} e il database per la bibliografia:
\href{http://mgrazzi.github.io/guida-tesi.tex}{file latex} e
\href{http://mgrazzi.github.io/bib_file_esempio.bib}{file bibtex}. Tra
i vari vantaggi di \LaTeX (\href{https://tobi.oetiker.ch/lshort/lshort.pdf}{a questo link}
trovate una guida), vi \`e la gestione completamente automatica della
bibliografia. Ovviamente potete anche usare altri programmi,


\bibliography{bib_file_esempio.bib}
%\printbibliography

\end{document}