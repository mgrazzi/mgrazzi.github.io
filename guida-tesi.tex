\documentclass[12pt]{article}
%\usepackage[a4paper]{geometry}
\usepackage[a4paper,left=1in,right=1in,top=1in,bottom=1in]{geometry}
\usepackage[italian,english]{babel}
\usepackage{amssymb}
\usepackage{amsmath}  %\documentstyle{numline} 
\usepackage{latexsym}
\usepackage{graphicx}
\usepackage{setspace}
\usepackage{rotating}
\usepackage{multirow}
\usepackage{verbatim}
\usepackage{subfigure}
\usepackage{authblk}
%\usepackage{fancyhdr}
\usepackage[usenames]{color}
\usepackage{soul}
\usepackage{setspace}
\usepackage{threeparttable}
\usepackage{caption}
\usepackage{dcolumn}
%\newcolumntype{.}{D{.}{.}{-1}}
\usepackage{placeins}
\usepackage{booktabs}

\usepackage[colorlinks,allcolors=blue]{hyperref}  % for buttons
%\usepackage[urlcolor=blue]{hyperref}  
% \hypersetup{
%    colorlinks=true,
%    linkcolor=blue,
%    filecolor=magenta,      
%    urlcolor=blue,
%}
%\onehalfspacing
\sethlcolor{Orange}
%\usepackage[first,bottom]{draftcopy}
\usepackage{draftwatermark}
\SetWatermarkText{Draft Version}
\SetWatermarkScale{5}
\usepackage{eurosym}
\usepackage[authoryear]{natbib}
%\bibliographystyle{chicago}
\bibliographystyle{ecta}
%\bibliographystyle{dcu}%{agsm}  %
\graphicspath{{./figures/}}

\renewcommand{\refname}{Bibliografia}

\hyphenation{in-te-res-sa-te   in-te-res-se   rap-pre-sen-ta-to   mo-del-li   mo-del-lo   si-ste-ma-ti-co   sem-pli-ci-sti-co   o-pe-ra-to-ri   me-de-si-me   i-ta-lia-no  ve-ri-fi-ca   sta-ti-sti-che   sta-ti-sti-ca-men-te   nor-ma-liz-za-re   con-si-de-ra-re   in-clu-sio-ne   e-sclu-sio-ne   e-sclu-de-re  o-pe-ra-zio-ni   o-pe-ra-zio-ne   sta-ti-sti-ca   ri-fe-ri-men-to   se-con-do  Se-con-do  de-via-zio-ne   va-rian-za  ela-sti-ca   con-si-de-ra-zio-ni mo-di-fi-che ma-nu-factu-ring Ma-nu-factu-rin Pro-dut-ti-vi-ta pro-dut-ti-vi-ta in-du-stria cam-bia-men-to am-mi-ni-stra-zio-ne}




\newcommand{\EP}{\mathrm{ \scriptscriptstyle EP}}
\newcommand{\SEP}{\mathrm{ \scriptscriptstyle SEP}}
\newcommand{\AEP}{\mathrm{ \scriptscriptstyle AEP}}
\newcommand{\GHYP}{\mathrm{ \scriptscriptstyle GHYP}}
\newcommand{\ASY}{\mathrm{ \scriptscriptstyle ASY}}
\newcommand{\RMSE}{\mathrm{ \scriptscriptstyle RMSE}}

\bibpunct{(}{)}{;}{a}{,}{,}


\title{ \textsc{Note per il lavoro di tesi}\thanks{Versione Febbraio
    2019. Ringrazio gli studenti che negli anni e nelle varie sedi con
    le loro domande hanno contribuito all stesura di queste pagine.}}


\author{Marco Grazzi\thanks{Universit\`a Cattolica del Sacro Cuore, Dipartimento di Politica Economica.}}

 % \small{\date{\today}}
\date{\empty}


\begin{document}
\maketitle





\section{Introduzione}

Queste note sono rivolte agli studenti che stanno valutando di
scrivere la tesi sotto la mia supervisione. Leggetele con attenzione
prima di venire a ricevimento o contattarmi per mail. Queste
indicazioni si aggiungono (e non sostituiscono!) alle istruzioni
fornite dalle segreterie studenti.

Il presente documento si rivolge sia agli studenti di laurea triennale
(LT) che di laurea magistrale (LM). A questi ultimi anticipo che, in
virt\`u del lavoro svolto in classe durante il corso, sar\`a richiesta
una parte di analisi empirica, preferibilmente di tipo
microeconomico. A tal fine gli studenti possono utilizzare le fonti
disponibili nella biblioteca tra le risorse elettroniche, che trovate
al link:
\href{https://biblioteche.unicatt.it/brescia-servizi-per-utenti-interni-ricerche-bibliografiche-e-banche-dati}{Risorse
  Elettroniche Unicatt}. Cliccate poi su Banche dati. Verificate se
sia necessario autenticarsi con le vostre credenziali.

%\subsubsection*{Scelta della materia e dell'argomento di tesi}

La tesi dovr\`a riguardare argomenti economici, in particolare quelli
affrontati nel corso. Questo vuol dire, solo per fare un esempio, che
se volete studiare le strategie di un'impresa o affrontare un caso
aziendale, questa non \`e la materia adeguata.

Se invece siete interessati ad una materia economica, ed in
particolare ad uno dei temi affrontati nel mio corso, allora il mio
(caloroso) suggerimento \`e di proporre voi uno o pi\`u argomenti di
tesi; avrete cos\`i occasione di scegliere e sviluppare un argomento di
vostro interesse. Al primo ricevimento discuteremo le vostre proposte.



\section{I tempi}
Siete ovviamenti voi stessi i responsabili del rispetto dei tempi
richiesti dall'amministra-zione.\footnote{Qui trovate il link con le
  scadenze:
  \href{https://milano.unicatt.it/facolta/economia-informazioni-per-gli-studenti-esami-di-laurea-294}{link
    scadenze}. } Considerate che in certi periodi mi pu\`o capitare di
ricever molte richieste di tesi e non sempre riesco a dare seguito a
tutte. Mi scuso in anticipo per questo.



\section{Lavoro preliminare: le fonti}\label{sec:fonti}
Una tesi, sia LT che LM, prevede un lavoro preliminare sulle fonti
bibliografiche. Cosa \`e gi\`a stato detto su un certo argomento?
Suggerisco di partire dai materiali del corso (libro, appunti,
articoli). Ci sono poi innumerevoli strumenti informatici che
permettono di espandere il raggio d'azione e risalire a fonti
precedenti o, al contrario, di spostarsi a lavori successivi che
citano quello di partenza; tra i molti richiamo
\textit{google.scholar}. Nel campo di ricerca inserite il titolo del
lavoro che cercate o parole chiavi del vostro argomento di
ricerca. Tra le fonti individuate date preferenza a quelle pi\`u
recenti e che sono gi\`a pubblicate su rivista in quanto hanno gi\`a
superato un vaglio editoriale. Segnalo inoltre inoltre le seguenti
riviste scientifiche che hanno un approccio rigoroso, ma hanno
carattere pi\`u divulgativo e quindi pi\`u facilmente accessibili:
\href{https://www.aeaweb.org/journals/jep}{Journal of Economic
  Perspective} e \href{https://www.aeaweb.org/journals/jel}{Journal of
  Economic Literature}.

Pu\`o essere utile perdersi per qualche giorno in questo ``mare'' di
fonti; \`e poi necessario concludere questa fase di esporazione e
delimitare il campo di analisi. Al termine di questo lavoro di
raccolta informazioni, selezione e lettura suggerisco di abbozzare la
struttura dell'elaborato.

Considerate infine che gran parte della bibliografia economica degli
anni pi\`u recenti \`e in lingua inglese.



\section{La struttura dell'elaborato}
Con estrema sintesi e semplificazione, un elaborato deve contenere
innanzitutto un'introduzione in cui si illustra la questione (la tesi,
la ``domanda di ricerca'') che sar\`a affrontata. La stessa
introduzione (o il capitolo successivo) deve fornire i riferimenti
alla letteratura scientifica di riferimento (andate alle fonti,
limitate al massimo o escludete proprio i riferimenti alla stampa
generalista). Per quanto recente pu\`o essere l'argomento che avete
scelto, qualcuno vi avr\`a gi\`a anticipato: prima di tentare di
aggiungere il vostro contributo fornite quindi una vostra
interpretazione dello \textit{stato dell'arte}.  I capitoli centrali
sviluppano il tema. Le conclusioni vi permettono di aggiungere il
vostro personale contributo interpretativo sull'argomento che avete
scelto.


\subsection*{L'analisi empirica (LM)}
Se la vostra \`e una tesi di LM uno dei capitoli centrali sar\`a
dedicato all'analisi empirica di uno o pi\`u settori economici con
dati micro, ovvero a livello di impresa. I dati li potete reperire
tramite le risorse online Unicatt (vedere a
\href{https://biblioteche.unicatt.it/brescia-servizi-per-utenti-interni-ricerche-bibliografiche-e-banche-dati}{questo}
link). Se tramite canali personali o lavorativi avete accesso a fonti
che ritenete pi\`u interessanti, valuteremo questa possibilit\`a ad un
ricevimento. 

Le analisi empiriche possono riguardare (in modo non esaustivo) i temi
gi\`a affrontati in classe. Qui potete trovare il riferimento a questo
visto durante il corso:\\
\href{http://mgrazzi.github.io/istruz-lavoro-gruppo.pdf}{mgrazzi.github.io/istruz-lavoro-gruppo.pdf}.




\section{Fonti bibliografiche}

Allo stesso modo con cui \`e importante fornire informazioni sul del
materiale utilizzato nell'elaborato (ad esempio, dati, tabelle,
figure), \`e necessario citare le fonti bibliografiche a cui si fa
riferimento. Tutte le fonti (articoli, libri, capitoli in libro,
working papers, etc) citate devono poi essere riportate nella
bibliografia in fondo alla tesi. Riporto di seguito un testo (in
inglese) che contiene citazioni a vari lavori, tutti poi riportati in
bibliografia. 

\vspace{0.3cm}

\small{An important stream of literature within industrial
  economics has for long been interested in assessing the contribution
  to employment creation stemming from the different firm-size
  classes. In this respect, at least since \cite{birch_1981}, small
  firms have been considered as a much relevant source of job
  creation. The increasing availability of firm level dataset has
  further contributed to foster research on the issue, starting from
  the seminal works of \cite{davis_haltiwanger_1992} and
  \cite{davis_etal_1996}. These studies represented a relevant
  advancement for the understanding of employment and industrial
  dynamics, in that they confirmed, by means of new methodological and
  empirical tools, that smaller firms are major players in terms of
  job churning, hence contributing both to employment creation and
  destruction \citep[among the others, also refer
  to][]{davis_haltiwanger_1995}.}



\subsection*{Quale programma utilizzare: la mia scelta}
Questo documento \`e stato scritto con \LaTeX. Ai seguenti link
trovate il file \textit{sorgente} e il database per la bibliografia:
\href{http://mgrazzi.github.io/guida-tesi.tex}{file latex} e
\href{http://mgrazzi.github.io/bib_file_esempio.bib}{file bibtex}. Tra
i vari vantaggi di \LaTeX
(\href{https://tobi.oetiker.ch/lshort/lshort.pdf}{a questo link}
trovate una guida), vi \`e la gestione completamente automatica della
bibliografia. Ovviamente potete anche usare altri programmi,


\bibliography{bib_file_esempio.bib}


\end{document}